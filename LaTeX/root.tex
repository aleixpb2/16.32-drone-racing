%%%%%%%%%%%%%%%%%%%%%%%%%%%%%%%%%%%%%%%%%%%%%%%%%%%%%%%%%%%%%%%%%%%%%%%%%%%%%%%%
%2345678901234567890123456789012345678901234567890123456789012345678901234567890
%        1         2         3         4         5         6         7         8

\documentclass[letterpaper, 10 pt, conference]{ieeeconf}  % Comment this line out if you need a4paper

%\documentclass[a4paper, 10pt, conference]{ieeeconf}      % Use this line for a4 paper

\IEEEoverridecommandlockouts                              % This command is only needed if you want to use the \thanks command

\overrideIEEEmargins                                      % Needed to meet printer requirements.

%In case you encounter the following error:
%Error 1010 The PDF file may be corrupt (unable to open PDF file) OR
%Error 1000 An error occurred while parsing a contents stream. Unable to analyze the PDF file.
%This is a known problem with pdfLaTeX conversion filter. The file cannot be opened with acrobat reader
%Please use one of the alternatives below to circumvent this error by uncommenting one or the other
%\pdfobjcompresslevel=0
%\pdfminorversion=4

% See the \addtolength command later in the file to balance the column lengths
% on the last page of the document

% The following packages can be found on http:\\www.ctan.org
%\usepackage{graphics} % for pdf, bitmapped graphics files
%\usepackage{epsfig} % for postscript graphics files
%\usepackage{mathptmx} % assumes new font selection scheme installed
%\usepackage{times} % assumes new font selection scheme installed
%\usepackage{amsmath} % assumes amsmath package installed
%\usepackage{amssymb}  % assumes amsmath package installed
\usepackage[]{graphicx}
\usepackage{balance}
\usepackage{hyperref}
\usepackage{float}
\usepackage{amsfonts}
\hypersetup{
	colorlinks   = true,  % Colour links instead of ugly boxes
	urlcolor     = black, % Colour for external hyperlinks
	linkcolor    = black, % Colour of internal links
	citecolor    = black  % Colour of citations
}
\usepackage{bm}

\title{\LARGE \bf
Optimal Control for Autonomous Drone Racing
}

\author{Aleix Paris$^{1}$% <-this % stops a space
\thanks{$^{1}$Graduate Research Assistant at the Aerospace Controls Laboratory,
        MIT, 77 Massachusetts Ave., Cambridge, MA, USA
        {\tt\small aleix@mit.edu}}%
}

\begin{document}

\maketitle
\thispagestyle{empty}
\pagestyle{empty}

%%%%%%%%%%%%%%%%%%%%%%%%%%%%%%%%%%%%%%%%%%%%%%%%%%%%%%%%%%%%%%%%%%%%%%%%%%%%%%%%
% TODO: use bm instead of boldsymbol
% TODO: cite GPOPS guide and files under "Literature"
% TODO: show all images in the folder
% TODO: MORE dynamics references
% TODO: make video, add to rar, and add as well the script and world used. Add README explaining that you need a ROS package (do you, actually?), and I used ACL's.
% TODO remove "we"
% TODO: nu eq too big
% TODO: read project description (stellar)
% TODO: read sheets, they have more TODOS
% TODO: TODOS in the code

\begin{abstract}

This paper studies the problem of optimal control of a quadrotor to minimize the time it takes it to pass trough several waypoints, that is, to finish a race.

\end{abstract}


%%%%%%%%%%%%%%%%%%%%%%%%%%%%%%%%%%%%%%%%%%%%%%%%%%%%%%%%%%%%%%%%%%%%%%%%%%%%%%%%
\section{INTRODUCTION}\label{s:intro}

Optimal control problems have been widely studied...
The Red Bull Air Race, where airplanes cross gates to end a circuit as fast as possible is an example of a similar problem that has been studied... Nowadays, every team has the role of a `tactician', which is in charge of...

MENTION PARKER AND AUTONOMOUS DRONE RACING COMPETITIONS...

The paper is structured as follows: Section~\ref{s:problem} presents the formulation of this problem, Section...


\section{PROBLEM FORMULATION}\label{s:problem}

Several authors have studied quadrotor dynamics \cite{IEEEexample:article_typical}. %TODO CITE
This section presents the assumptions presumed, the geometry of a quadrotor, its dynamics, the state-space model used, and the mathematical formulation of the problem in hand.

\subsection{Assumptions}

The formulation that follows considers the following assumptions:
\begin{enumerate}
	\item The maximum and minimum altitudes of the quadrotor are similar and thus the air density and gravity are constants.
	\item The vehicle is flying in zero-wind conditions.
	\item No ground effect is considered.
	\item \label{as:Ir} The angular velocities of the rotors are similar, and the rotors' inertia is small.
	\item \label{as:inertia} The quadrotor is symmetrical with its four arms aligned with the body x- and y-axes.
	\item The propellers are rigid and thus blade flapping does not occur.
\end{enumerate}

\subsection{Quadrotor Geometry and Notation}

The quadrotor's absolute linear position is defined in the inertial frame with the vector \bm{$\xi$}. Similarly, the attitude (angular position of the drone with respect to the inertial frame) is defined with the vector \bm{$\eta$}. Roll angle $\phi$ determines the rotation of the vehicle around the x-axis, pitch angle $\theta$ defines a rotation around the y-axis, and yaw angle $\psi$ determines the quadrotor's rotation around the z-axis:


$$\bm{\xi}=\left[ \begin{array}{l}{x} \\ {y} \\ {z}\end{array}\right],
\quad \bm{\eta}=\left[ \begin{array}{l}{\phi} \\ {\theta} \\ {\psi}\end{array}\right]$$

The origin of the body frame, indicated with a B, is the center of mass of the quadrotor. In this frame, the vehicle's linear velocities \bm{$V_B$} and angular velocities \bm{$\nu$} are:

$$\bm{V}_{B}=\left[ \begin{array}{c}{v_{x, B}} \\ {v_{y, B}} \\ {v_{z, B}}\end{array}\right], \quad \bm{\nu}=\left[ \begin{array}{l}{p} \\ {q} \\ {r}\end{array}\right]$$

% TODO: delete VB?

Figure~\ref{fig:quad_frame} shows the inertial and body frame, as well as the Euler and angular velocity angles defined previously.

\begin{figure}[!htpb]
	\centering
	\includegraphics[width=1.0\linewidth]{Images/quad_frame.png}
	\caption{Inertial and body-fixed frame of the quadrotor, showing the Euler and angular velocity angles [?] %TODO CITE
		}
	\label{fig:quad_frame}
\end{figure}


 The rotation from the body frame to the inertial frame can be defined with the matrix

$$\bm{R}=\left[ \begin{array}{ccc}{C_{\psi} C_{\theta}} & {C_{\psi} S_{\theta} S_{\phi}-S_{\psi} C_{\phi}} & {C_{\psi} S_{\theta} C_{\phi}+S_{\psi} S_{\phi}} \\ {S_{\psi} C_{\theta}} & {S_{\psi} S_{\theta} S_{\phi}+C_{\psi} C_{\phi}} & {S_{\psi} S_{\theta} C_{\phi}-C_{\psi} S_{\phi}} \\ {-S_{\theta}} & {C_{\theta} S_{\phi}} & {C_{\theta} C_{\phi}}\end{array}\right]$$

in which $S_{x}=\sin (x)$ and $C_{x}=\cos (x)$. Note that this rotation matrix is orthogonal and thus the rotation matrix from the inertial frame to the body frame is $\bm{R^{-1} = \bm{R^T}}$.

To obtain the angular velocities in the body frame from the angular velocities in the inertial frame, the matrix $\bm{W_\eta}$ should be used:

$\bm{\nu}=\bm{W}_{\eta} \dot{\bm{\eta}}, \hfill \left[ \begin{array}{l}{p} \\ {q} \\ {r}\end{array}\right]=\left[ \begin{array}{ccc}{1} & {0} & {-S_{\theta}} \\ {0} & {C_{\phi}} & {C_{\theta} S_{\phi}} \\ {0} & {-S_{\phi}} & {C_{\theta} C_{\phi}}\end{array}\right] \left[ \begin{array}{c}{\dot{\phi}} \\ {\dot{\theta}} \\ {\dot{\psi}}\end{array}\right]$

Its inverse is the transformation matrix from the body frame to the inertial frame, which will be used later to assemble the quadrotor dynamics:

$\dot{\bm{\eta}}=\bm{W}_{\eta}^{-1} \bm{\nu}$, \hfill $\left[ \begin{array}{c}{\dot{\phi}} \\ {\dot{\theta}} \\ {\dot{\psi}}\end{array}\right]=\left[ \begin{array}{ccc}{1} & {S_{\phi} T_{\theta}} & {C_{\phi} T_{\theta}} \\ {0} & {C_{\phi}} & {-S_{\phi}} \\ {0} & {S_{\phi} / C_{\theta}} & {C_{\phi} / C_{\theta}}\end{array}\right] \left[ \begin{array}{l}{p} \\ {q} \\ {r}\end{array}\right]$

where $T_{x}=\tan (x)$.

As shown in Figure~\ref{fig:quad_frame} and stated in assumption~\ref{as:inertia}, the drone is symmetric with the arms aligned with the body axes. Therefore, the inertia matrix $\textbf{I}$ is diagonal, and $I_{xx} = I_{yy}$:

 $$\boldsymbol{I}=\left[ \begin{array}{ccc}{I_{x x}} & {0} & {0} \\ {0} & {I_{y y}} & {0} \\ {0} & {0} & {I_{z z}}\end{array}\right]$$
 
 \subsection{Quadrotor Dynamics}

The force $f_i$ created by the angular velocity of rotor $i$, denoted with $\omega_i$, in the direction of the rotor axis is:

$$f_{i}=k \omega_{i}^{2}$$

Additionally, torque $\tau_{M_i}$ is created around the rotor axis:

$$\tau_{M_{i}} = b \omega_{i}^{2} + I_{r} \dot{\omega}_{i} $$

where the lift constant is $k$, the drag constant is $b$ and the inertia moment of the rotor is $I_{r}$. As assumption~\ref{as:Ir} considered, the rotor's inertia is small, and $\dot{\omega}_{i}$ is also usually small. Therefore, this term can be omitted.

The rotors create thrust $T$ in the direction of the body z-axis, and torque $\bm{\tau_B}$ consists of torques $\tau_\phi, \tau_\theta, \tau_\psi$ in the direction of the corresponding body frame angles:

$$T=\sum_{i=1}^{4} f_{i}=k \sum_{i=1}^{4} \omega_{i}^{2}, \quad \quad \boldsymbol{T}^{B}=\left[ \begin{array}{c}{0} \\ {0} \\ {T}\end{array}\right]$$

$$
\boldsymbol{\tau}_{B}=\left[ \begin{array}{c}{\tau_{\phi}} \\ {\tau_{\theta}} \\ {\tau_{\psi}}\end{array}\right]=
\left[ \begin{array}{c}{l k\left(-\omega_{2}^{2}+\omega_{4}^{2}\right)} \\ {l k\left(-\omega_{1}^{2}+\omega_{3}^{2}\right)} \\ {\sum_{i=1}^{4} \tau_{M_{i}}}\end{array}\right]
$$

in which l is the distance between the rotor and the center of mass of the quadcopter.

The Newton-Euler equations for the quadrotor are:

$$m \ddot{\boldsymbol{\xi}}=\boldsymbol{G}+\boldsymbol{R} \boldsymbol{T}_{B}$$

$$\left[ \begin{array}{c}{\ddot{x}} \\ {\ddot{y}} \\ {\ddot{z}}\end{array}\right]=-g \left[ \begin{array}{l}{0} \\ {0} \\ {1}\end{array}\right]+\frac{T}{m} \left[ \begin{array}{c}{C_{\psi} S_{\theta} C_{\phi}+S_{\psi} S_{\phi}} \\ {S_{\psi} S_{\theta} C_{\phi}-C_{\psi} S_{\phi}} \\ {C_{\theta} C_{\phi}}\end{array}\right]$$

where g is Earth's gravity, 9.81 $m/s^2$. Additionally:
 $$\boldsymbol{I} \dot{\boldsymbol{\nu}}+\boldsymbol{\nu} \times(\boldsymbol{I} \boldsymbol{\nu})+\mathbf{\Gamma}=\boldsymbol{\tau_B}$$
 
 $$\boldsymbol{\dot{\nu}}=\boldsymbol{I}^{-1}\left(-\left[ \begin{array}{c}{p} \\ {q} \\ {r}\end{array}\right] \times \left[ \begin{array}{c}{I_{x x} p} \\ {I_{y y} q} \\ {I_{z z} r}\end{array}\right]-I_{r} \left[ \begin{array}{c}{p} \\ {q} \\ {r}\end{array}\right] \times \left[ \begin{array}{c}{0} \\ {0} \\ {1}\end{array}\right] \omega_{\Gamma}+\boldsymbol{\tau_B}\right)$$
 
 $$
 \begin{array}{c}
 \left[ \begin{array}{c}{\dot{p}} \\ {\dot{q}} \\ {\dot{r}}\end{array}\right]= 
 \left[ \begin{array}{c}{\left(I_{y y}-I_{z z}\right) q r / I_{x x}} \\ {\left(I_{z z}-I_{x x}\right) p r / I_{y y}} \\ {(I_{x x}-I_{y y}) p q / I_{z z}}\end{array}\right] 
 - I_{r} \left[ \begin{array}{c}{q / I_{x x}} \\ {-p / I_{y y}} \\ {0}\end{array}\right] \omega_{\Gamma} \\
 + \left[ \begin{array}{c}{\tau_{\phi} / I_{x x}} \\ {\tau_{\theta} / I_{y y}} \\ {\tau_{\psi} / I_{z z}}\end{array}\right]
 \end{array}
 $$
 
 where $\omega_\Gamma = \omega_1 - \omega_2 + \omega_3 - \omega_4$. By assumption~\ref{as:Ir}, the term that includes $I_r$ and $\omega_\Gamma$ can be omitted.
 
 Finally, this work considers aerodynamic drag caused by the vehicle's translation. Therefore, the revised Netwon equation is:
 
  $$
  \begin{array}{l}
  \left[ \begin{array}{l}{\ddot{x}} \\ {\ddot{y}} \\ {\ddot{z}}\end{array}\right]=-g \left[ \begin{array}{l}{0} \\ {0} \\ {1}\end{array}\right]+\frac{T}{m} \left[ \begin{array}{c}{C_{\psi} S_{\theta} C_{\phi}+S_{\psi} S_{\phi}} \\ {S_{\psi} S_{\theta} C_{\phi}-C_{\psi} S_{\phi}} \\ {C_{\theta} C_{\phi}}\end{array}\right] \\
  \quad \quad \quad \quad  - \frac{1}{m} \left[ \begin{array}{ccc}{A_{x}} & {0} & {0} \\ {0} & {A_{y}} & {0} \\ {0} & {0} & {A_{z}}\end{array}\right] \left[ \begin{array}{c}{\dot{x}} \\ {\dot{y}} \\ {\dot{z}}\end{array}\right]
  \end{array}
  $$
  
  where $A_x, A_y$, and $A_z$ are the drag force coefficients for velocities in the corresponding directions of the inertial frame.
 
\subsection{State-space Representation}

The state of the quadrotor can be represented as follows:

 $$ \bm{X} = \left[ 
 \begin{array}{c}
  {\bm{x}}  \\
  {\bm{\dot{x}}}  \\
  {\bm{\eta}}  \\
  {\bm{\nu}}
  \end{array}\right]$$

that is, a column vector of $4 \cdot 3 = 12$ components that contains the inertial position, the inertial velocity, the Euler angles and the angular velocities in the body frame. The control inputs are:

 $$ \bm{U} = \left[ 
 \begin{array}{c}
 {T}  \\
 {\tau_\phi}  \\
 {\tau_\theta}  \\
 {\tau_\psi}
 \end{array}\right]$$
 
 that is, the total thrust $T$ and the torques $\tau$ that cause a roll, pitch, and yaw angle change, respectively.
 
 The dynamics previously derived can be assembled in vector $\bm{\dot{X}} = f(\boldsymbol{X}, \boldsymbol{U})$, which represents the change of the state as a function of the state and the inputs. That is:
 
 $$
 \dot{\boldsymbol{X}} =
 \left[
 \begin{array}{c}
 \dot{x} \\
 {}\\
 \dot{y} \\
 {}\\
 \dot{z} \\
 {}\\
 {}\\
 \ddot{x} \\
 {}\\
 \ddot{y} \\
 {}\\
 \ddot{z} \\
 {}\\
 {}\\
 \dot{\phi} \\
 {}\\
 \dot{\theta} \\
 {}\\
 \dot{\psi} \\
 {}\\
 {}\\
 \dot{p} \\
 {}\\
 \dot{q} \\
 {}\\
 \dot{r}
 \end{array}
 \right]
 =
 \left[
 \begin{array}{l}
 \dot{x} \\
 {}\\
 \dot{y} \\
 {}\\
 \dot{z} \\
 {}\\
 {}\\
 \frac{T}{m}[C_{\psi} S_{\theta} C_{\phi} + S_{\psi} S_{\phi}] - \frac{A_x}{m} \dot{x} \\
 {}\\
 \frac{T}{m}[ S_{\psi} S_{\theta} C_{\phi} - C_{\psi} S_{\phi}] - \frac{A_y}{m} \dot{y} \\
 {}\\
 -g+\frac{T}{m}[C_{\theta} C_{\phi}] - \frac{A_z}{m} \dot{z} \\
 {}\\
 {}\\
 p+ q[S_{\phi} T_{\theta}] + r[C_{\phi} T_{\theta}] \\
 {}\\
 q[C_{\phi}]-r[S_{\phi}] \\
 {}\\
 q \frac{S_{\phi}}{C_{\theta}} + r \frac{C_{\phi}}{C_{\theta}} \\
 {}\\
 {}\\
 \frac{I_{yy}-I_{zz}}{I_{xx}} q r  +\frac{\tau_{\phi}}{I_{xx}} \\
 {}\\
 \frac{I_{zz}-I_{xx}}{I_{yy}} p r +\frac{\tau_{\theta}}{I_{yy}} \\
 {}\\
 \frac{I_{xx}-I_{yy}}{I_{zz}} p q +\frac{\tau_{\psi}}{I_{zz}}
 \end{array}
 \right]
 $$

\subsection{Mathematical Formulation}\label{subs:math_form}
The problem in hand consists of minimizing the time a drone takes to complete a race, that is, to cross all the gates in a specific order as fast as possible. In this paper, the gates are considered point constraints only, and it does not matter the direction of the velocity vector. This can be formulated mathematically as an optimal control problem:

\bigskip

Minimize \quad $\int_{t_0}^{t_f} 1 dt = \int_{0}^{t_f} 1 dt = t_f$

\bigskip

subject to:

$$ \bm{\dot{X}} = f(\boldsymbol{X}, \boldsymbol{U}) $$

$$ \bm{X}(0) = 0 $$

\quad $ \bm{x}(t_i) = \bm{g_i},$ \quad $t_i < t_{i+1} $ for $i = 1 .. n_{gates}-1$

$$ z \ge 0 $$

$$ 0 \le T \le T_{max} $$
$$  |\tau_{\phi}| \le \tau_{\phi,max} $$
$$  |\tau_{\theta}| \le \tau_{\theta,max} $$
$$  |\tau_{\psi}| \le \tau_{\psi,max} $$

The first constraint imposes the quadrotor dynamics. The second one, states that the initial state of the drone is 0, that is, the drone is stopped in the origin and the body axes match the global axes. The third constraint are the gates: the drone's position $\bm{x}$ has to match the gates' positions $\bm{g_i}$ in increasing times (that is, the gates have to be crossed in order, from number 1 to number $n_{gates}$). The next constraint ensures that the drone will not crash against the ground. The last four constraints limit the control inputs, necessary to account for the limitations of the quadrotor's propellers and motors.

\section{PROBLEM IMPLEMENTATION}\label{s:impl}

The problem was implemented in GPOPS-II \cite{patterson2014gpops}, a MATLAB software to solve optimal control problems. As shown in GPOPS-II's User Guide \cite{gpops_guide}, a program must have three main functions: the main, the continuous and the endpoint function, which are explained in the following subsections.

To specify the gate constraints, the program is divided in $n_{gates}$ phases. The first phase is from the initial state (where all the states are 0) to the first gate, the second phase is from the end of the first phase to the second gate, and so on.

\subsection{Main Function}
In this function, the race and drone parameters are loaded, the program is set up and the bounds and guesses are specified:
\begin{itemize}
	\item \textbf{Bounds:} the bounds limit the states and controls allowed for each phase. The time, position, velocity, orientation and rates are limited to realistic values, although the quadrotor could technically surpass them. For example, this work does not allow the drone to fly upside down or faster than 50 $m/s$. As for the control inputs, they are limited to the allowed values as shown in Subsection~\ref{subs:math_form}.
	
	Additionally, the z is limited to values greater than 0 to avoid the drone crashing into the ground, the initial state is set to 0, the difference of states in contiguous phases is set to 0 and the final position for each phase is set to the corresponding gate.
	\item \textbf{Guesses:} they set values to the time, state and control for each phase to help the algorithm converge.
\end{itemize}

Finally, the data is saved and plotted using auxiliary functions.

\subsection{Continuous Function}

This function contains all the dynamics explained in Section~\ref{s:problem}. For every phase, $\bm{\dot{X}}$ is computed and returned. Additionally, the path constraint variable is returned, that is, the state $z$. As mentioned before, its minimum is set to 0 using the lower and upper path constraint bounds.

\subsection{Endpoint Function}

The endpoint function returns the objective function, which is the final time of the last phase, and also returns the difference of the states and times between contiguous phases as something called `eventgroup'. This difference is set to 0 by the bounds, like previously explained.

\subsection{Video Functions}

To make the video of the drone performing the race...


\section{RESULTS}\label{s:res}

\subsection{Drone Parameters}
A

\subsection{Experiments}

Show simple experiment with/out considering z constraint. Show gates.

Show gates and plots of final experiment.

The video can be found in ...

\section{CONCLUSIONS}\label{s:conclusions}

Copy abstract... Future work could impose additional gate constraints such as limiting the velocity to a cone, although this constraint is not noticeable if the gates are positioned one (more or less) in front of the other as would happen in a real race.

\addtolength{\textheight}{-12cm}   % This command serves to balance the column lengths
                                  % on the last page of the document manually. It shortens
                                  % the textheight of the last page by a suitable amount.
                                  % This command does not take effect until the next page
                                  % so it should come on the page before the last. Make
                                  % sure that you do not shorten the textheight too much.

%%%%%%%%%%%%%%%%%%%%%%%%%%%%%%%%%%%%%%%%%%%%%%%%%%%%%%%%%%%%%%%%%%%%%%%%%%%%%%%%
%%%%%%%%%%%%%%%%%%%%%%%%%%%%%%%%%%%%%%%%%%%%%%%%%%%%%%%%%%%%%%%%%%%%%%%%%%%%%%%%
\section*{APPENDIX}

Nothing...

%%%%%%%%%%%%%%%%%%% CHECK BIB MANUAL %%%%%%%%%%%%%%%%%%%%%%%%%%%%%%%%%%%%%%%%%%%%%%%%%%%%%%%%%%%%%%%%%
%\begin{thebibliography}{99}

\balance

\makeatletter
\def\endthebibliography{%
	\def\@noitemerr{\@latex@warning{Empty `thebibliography' environment}}%
	\endlist
}
\makeatother

\bibliographystyle{unsrt}
\bibliography{IEEEexample}
%\end{thebibliography}
%%%%%%%%%%%%%%%%%%%%%%%%%%%%%%%%%%%%%%%%%%%%%%%%%%%%%%%%%%%%%%%%%%%%%%%%%%%%%%%%%%%%%%%%%%%%%%%%%%%%%%

\end{document}
